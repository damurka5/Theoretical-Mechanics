\documentclass[a4paper,11pt,oneside,article]{memoir}
% lidt marginer
\setlrmarginsandblock{3cm}{*}{1}
\setulmarginsandblock{3cm}{*}{1}
\setheadfoot{2cm}{\footskip}          
\checkandfixthelayout[nearest]

\setlength{\parindent}{0pt}

\usepackage[utf8]{inputenc}
\usepackage[danish]{babel}
\renewcommand\danishhyphenmins{22}
\usepackage[T1]{fontenc}
\usepackage{lmodern}
\usepackage{icomma}
\usepackage{gensymb}
\usepackage{bm, amsmath,mathdots, amssymb, mathtools}
\usepackage[normalem]{ulem}
\usepackage{array, booktabs, tabularx}
\usepackage{graphicx, caption, subfig, xcolor}


\captionsetup{font=small,labelfont=bf}
\graphicspath{ {./images/} }

\makepagestyle{mypagestyle}
\copypagestyle{mypagestyle}{empty}
\makeoddhead{mypagestyle}{Nurtdinov Damir 31.05.2002\\d.nurtdinov@innopolis.university}{\quad}{{September 2022}\\Theoretical Mechanics: HW1}

\makeheadrule{mypagestyle}{\textwidth}{\normalrulethickness}
\makeoddfoot{mypagestyle}{}{\thepage}{}

\pagestyle{mypagestyle}

\usepackage{enumerate}

\usepackage[output-decimal-marker={,}]{siunitx}

\usepackage[hidelinks=true]{hyperref}
\title{Theoretical Mechanics HW1}
\author{Дамир Нуртдинов}
\date{September 2022}

\def\doubleunderline#1{\underline{\underline{#1}}}

\begin{document}

\section*{Task 1}
\subsection{Problem:}
$OM = s_r(t) = 6 \pi t^2$\\
$\phi (t) = \frac{\pi t^3}{6}$\\
(1) Simulate the mechanism\\
(2) Find absolute, transport and relative velocities and accelerations for $M$\\
(3) Find $t$, when $M$ reaches $O$ \\
(4) Put on simulation the previous statements



The universe is immense and it seems to be homogeneous, 
in a large scale, everywhere we look at.

\includegraphics{images/vel_coord.png}


There's a picture of a galaxy above

\subsection{Solution:}
I denoted the center of $OA$ as $C$\\
(1) and (4)  The simulation is available by the\\ Link
\url{https://www.geogebra.org/m/hhxbcrb6}\\
(2) \begin{equation}
\begin{split}
    \omega = \dot \phi(t)\\
    \doubleunderline{
    V_M ^{tr} = \omega O_2A, V_M ^{tr} \perp O_2A
    }\\
    \doubleunderline{
    V_M ^{rel} = \dot s_r(t) = 12\pi t, V_M ^{rel} \perp CM
    }\\
        \doubleunderline{
    \overrightarrow{V_M} = \overrightarrow{V_M ^{tr}} + \overrightarrow{V_M ^{rel}}
    }\\
\end{split}
\end{equation}
The logic for accelerations is the same. There is no Coriolis acceleration, because semicirlce $D$ does not have angular velocity ($\overrightarrow{V_A} = \overrightarrow{V_O}$)\\
\begin{equation}
\begin{split}
    \doubleunderline{
    a_{tr} ^{n} = \omega ^2 O_2A,  \overrightarrow{a_{tr} ^{n}} || O_2A
    }\\
    \epsilon = \ddot \phi = \pi t
    \\
    \doubleunderline{
    a_{tr} ^{\tau} = \epsilon O_2A, \overrightarrow{a_{tr} ^{\tau}} || \overrightarrow{V_{tr}}
    }\\
    \overrightarrow{a_M ^{rel}} =\overrightarrow{a_{rel} ^{n}} + \overrightarrow{a_{rel} ^{\tau}}\\
    \doubleunderline{
    a_{rel} ^{n} = \frac{V_{rel}^2}{R},  \overrightarrow{a_{rel} ^{n}} || CM
    }\\
    \doubleunderline{
    a_{rel} ^{\tau} = \ddot s_r = 12\pi, \overrightarrow{a_{rel} ^{\tau}} || \overrightarrow{V_{rel}}
    }\\
    \doubleunderline{
     \overrightarrow{a_M ^{tr}} =\overrightarrow{a_{tr} ^{n}} + \overrightarrow{a_{tr} ^{\tau}}
    }
\end{split}
\end{equation}

(3) Here I just solved a simple equation:
\begin{equation}
\begin{split}
    s_r (t) = \phi (t) R\\
    6 \pi t^2 = \frac{\pi t^3}{6} R\\
    \doubleunderline{
    t = \sqrt{\frac{R}{6}}
    }
\end{split}
\end{equation}


\section*{Task 2}
\subsection{Problem:}
$OM = s_r(t) = 75 \pi (0.1 + 0.3t^2)$\\
$\phi (t) = 2t - 0.3t^2$\\
(1) Simulate the mechanism\\
(2) Find absolute, transport and relative velocities and accelerations for $M$\\
(3) Find $t$, when $M$ reaches $O$ second time\\
(4) Put on simulation the previous statements
\subsection{Solution:}


I denoted the center of the circle as $O_2$\\
(1) and (4)  The simulation is available by the\\ Link
\url{https://www.geogebra.org/m/pzvphvd3}\\
(2) \begin{equation}
\begin{split}
    \omega = \dot \phi(t)\\
    \doubleunderline{
    V_M ^{tr} = \omega O_1M, V_M ^{tr} \perp O_1M
    }\\
    \doubleunderline{
    V_M ^{rel} = \dot s_r(t) = 75\pi (0.1+0.6t), V_M ^{rel} \perp O_2M
    }\\
        \doubleunderline{
    \overrightarrow{V_M} = \overrightarrow{V_M ^{tr}} + \overrightarrow{V_M ^{rel}}
    }\\
\end{split}
\end{equation}
The logic for accelerations is the same. There is Coriolis acceleration\\
\begin{equation}
\begin{split}
    \doubleunderline{
    a_{tr} ^{n} = \omega ^2 O_1M,  \overrightarrow{a_{tr} ^{n}} || O_1M
    }\\
    \epsilon = \ddot \phi = -0.6
    \\
    \doubleunderline{
    a_{tr} ^{\tau} = \epsilon O_1M, \overrightarrow{a_{tr} ^{\tau}} || \overrightarrow{V_{tr}}
    }\\
    \overrightarrow{a_M ^{rel}} =\overrightarrow{a_{rel} ^{n}} + \overrightarrow{a_{rel} ^{\tau}}\\
    \doubleunderline{
    a_{rel} ^{n} = \frac{V_{rel}^2}{R},  \overrightarrow{a_{rel} ^{n}} || O_2M
    }\\
    \doubleunderline{
    a_{rel} ^{\tau} = \ddot s_r = 75\pi 0.6t, \overrightarrow{a_{rel} ^{\tau}} || \overrightarrow{V_{rel}}
    }\\
    \doubleunderline{
    \overrightarrow{a_M ^{cor}} = 2 \omega \times  \overrightarrow{V_{rel}}
    }
    \\
    \doubleunderline{
     \overrightarrow{a_M ^{tr}} =\overrightarrow{a_{tr} ^{n}} + \overrightarrow{a_{tr} ^{\tau}} + \overrightarrow{a_{M} ^{cor}}
     }
\end{split}
\end{equation}

(3) Here I just solved a simple equation:
\begin{equation}
\begin{split}
    s_r (t) = \phi (t) R\\
    75 \pi (0.1 + 0.3t^2) = 2\pi R\\
    \doubleunderline{
    t = \frac{-15+\sqrt{225+720R}}{90} \approx 1.17704 s
    }
\end{split}
\end{equation}
\end{document}
