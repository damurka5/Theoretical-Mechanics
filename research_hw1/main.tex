\documentclass[a4paper,11pt,oneside,article]{memoir}
% lidt marginer
\setlrmarginsandblock{3cm}{*}{1}
\setulmarginsandblock{3cm}{*}{1}
\setheadfoot{2cm}{\footskip}          
\checkandfixthelayout[nearest]

\setlength{\parindent}{0pt}

\usepackage[utf8]{inputenc}
\usepackage[danish]{babel}
\renewcommand\danishhyphenmins{22}
\usepackage[T1]{fontenc}
\usepackage{lmodern}
\usepackage{icomma}
\usepackage{gensymb}
\usepackage{bm, amsmath,mathdots, amssymb, mathtools}
\usepackage[normalem]{ulem}
\usepackage{array, booktabs, tabularx}
\usepackage{graphicx, caption, subfig, xcolor}


\captionsetup{font=small,labelfont=bf}
\graphicspath{ {./images/} }

\makepagestyle{mypagestyle}
\copypagestyle{mypagestyle}{empty}
\makeoddhead{mypagestyle}{Nurtdinov Damir 31.05.2002\\d.nurtdinov@innopolis.university}{\quad}{{September 2022}\\Theoretical Mechanics: HW4}

\makeheadrule{mypagestyle}{\textwidth}{\normalrulethickness}
\makeoddfoot{mypagestyle}{}{\thepage}{}

\pagestyle{mypagestyle}

\usepackage{enumerate}

\usepackage[output-decimal-marker={,}]{siunitx}

\usepackage[hidelinks=true]{hyperref}
\title{Theoretical Mechanics HW1}
\author{Дамир Нуртдинов}
\date{September 2022}

\def\doubleunderline#1{\underline{\underline{#1}}}

\begin{document}

\section*{Task 1}
\subsection{Problem:}
Determine reaction forces and the forces in the interim pins of the composite stud. $P_1 = 12$, $P_2 = 18$, $M_1 = 36$,$q = 1.4$ 
\subsection{Solution:}
"For the system to be in static equilibrium, there must be three rods: AC, CE, EF" – A. Maloletov.\\
\underline{Research object:} system of 3 rods: AC, CE, EF\\
Here is the picture:\\
\includegraphics[width=15cm]{images/pic1.png}\\
\underline{Method:} static\\
\underline{Body is fixed:}\\
A – Roller support, $R_A$ - perpendicular to it floor \\
B – Pin connection, $R_x$, $R_y$\\
C – Rotational joint, $R_x$, $R_y$\\
D – Roller support, $R_y$\\
E – Rotational joint, $R_x$, $R_y$\\
F – Roller support, $R_y$\\
The next step was to draw the reactions:\\
\includegraphics[width=15cm]{images/pic2.jpg}\\
Where $Q_1 = 3q$, $Q_2 = 2.5q$, $Q_3 = 2q$ – representation of a distributed force as a 3 forces. There is no $R^B_x$ because we consider AC as one rigid body, it has internal horizontal reactions, but they compensate each other. \\
\underline{Force analysis:}\\
Known: $P_1$, q, M1, $P_2$\\
Unknown: $R_A$, $R^B_y$, $R^C_x$, $R^C_y$, $R^D_y$, $R^E_x$, $R^E_y$, $R^F_y$.\\
\underline{Solution:}\\
I started from stud EF:\\
\begin{equation}
    \begin{cases}
    ox: R^E_x - P_2 cos(60\degree) = 0\\
    oy: R^E_y - P_2 sin(60\degree) + R^F_y\\
    M_E: -M_1 - P_2 sin(60\degree) 2.5 + R^F_y 4.5 = 0
     \end{cases}
\end{equation}
From the system above I got:
\begin{equation}
    \begin{split}
    \doubleunderline{R^F_y \approx 16.66025}\\
    \doubleunderline{R^E_y \approx -1.07179}\\
    \doubleunderline{R^E_x =9}
     \end{split}
\end{equation}

The stud CE:
\begin{equation}
    \begin{cases}
    ox: R^{'C}_x - R^{'E}_x = 0\\
    oy: - R^{'C}_y - Q_2 + R^D_y - Q_3 - R^{'F}_y = 0\\
    M_C: -Q_2 \frac{2.5}{2}+ R^D_y 2.5 - Q_3 3.5 - R^{'E}_y 4.5 = 0
     \end{cases}
\end{equation}
$|R^{'C}_x| = |R^{C}_x|$ ,$|R^{'C}_y| = |R^{C}_y|$  \\
From the system (3) I got:
\begin{equation}
    \begin{split}
    \doubleunderline{R^D_y \approx 3.740778}\\
    \doubleunderline{R^C_x = 9}\\
    \doubleunderline{R^C_y \approx -1.487432}
     \end{split}
\end{equation}

The stud AC:
\begin{equation}
    \begin{cases}
    ox: R_A cos(45\degree) - R^{C}_x = 0\\
    oy: R_A sin(45\degree - P_1 + R^B_y - Q_1 + R^C_y = 0\\
    M_A: -P_1 2 + R^B_y 4 - Q_1 5.5 + R^C_y 7 =  0
     \end{cases}
\end{equation}

From the system (5) I got:
\begin{equation}
    \begin{split}
    \doubleunderline{R^B_y \approx -14.378006}\\
    \doubleunderline{R_A = 12.7279}\\
     \end{split}
\end{equation}

\section*{Task 2}
\subsection{Problem:}
Determine the reaction forces in rods supporting a  thin horizontal rectangular plate of weight G under action of force P applied along the side AB.\\
$G = 18$,$P=30$, $a=4$, $b=4.5$, $c=3.5$
\subsection{Solution:}
\underline{Research object:} system of rods and horizontal rectangular plate.
\underline{Method:} static\\
\underline{Body is fixed:}\\
B – Pin connection, $R_x$, $R_y$, $R_z$\\
C – Pin connection, $R_x$, $R_y$, $R_z$\\
D – Pin connection, $R_x$, $R_y$, $R_z$\\
4 bottom supports with pin connections, $R_x$, $R_y$, $R_z$\\
Here is the picture:\\
\includegraphics[width=15cm]{images/pic3.jpg}\\
\underline{Force analysis:}\\
Known: $P$, $G$\\
Unknown: $S_1$,$S_2$,$S_3$,$S_4$,$S_5$,$S_6$.\\
\underline{Solution:}\\
I wrote equation of moments around axis and equation of forces:
\begin{equation}
    \begin{cases}
    M_{ix}:  S_3 b + S_4 sin(\alpha) b + S_5 b + G \frac{b}{2} = 0\\
    M_{iy}: S_5 a + G\frac{a}{2}= 0 \\
    M_{iz}: P a - S_4cos(\alpha) b = 0\\
    ox: -S_6 cos(\alpha) - S_4 cos(\alpha) = 0\\
    oy: S_2 cos(\beta) + P = 0\\
    oz: -S_1 - S_6 sin(\alpha) - S_2 sin(\beta) - S_3 - S_4 sin(\alpha) - S_5- S_6 = 0\\
    sin(\alpha) = \frac{c}{\sqrt{a^2+c^2}}\\
    sin(\beta) = \frac{c}{\sqrt{b^2+c^2}}\\
    cos(\beta) = \frac{b}{\sqrt{b^2+c^2}}\\
    \end{cases}
\end{equation}

From this system I got:
\begin{equation}
    \begin{split}
    \doubleunderline{S_1 \approx -14.7710257}\\
    \doubleunderline{S_2 \approx 38.00586}\\
    \doubleunderline{S_3 \approx -17.56233}\\
    \doubleunderline{S_4 \approx 26.67}\\
    \doubleunderline{S_5  = -9}\\
    \doubleunderline{S_6 \approx -26.67}\\
     \end{split}
\end{equation}

\end{document}