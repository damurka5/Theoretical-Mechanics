\documentclass[a4paper,11pt,oneside,article]{memoir}
% lidt marginer
\setlrmarginsandblock{3cm}{*}{1}
\setulmarginsandblock{3cm}{*}{1}
\setheadfoot{2cm}{\footskip}          
\checkandfixthelayout[nearest]

\setlength{\parindent}{0pt}

\usepackage[utf8]{inputenc}
\usepackage[danish]{babel}
\renewcommand\danishhyphenmins{22}
\usepackage[T1]{fontenc}
\usepackage{lmodern}
\usepackage{icomma}
\usepackage{gensymb}
\usepackage{bm, amsmath,mathdots, amssymb, mathtools}
\usepackage[normalem]{ulem}
\usepackage{array, booktabs, tabularx}
\usepackage{graphicx, caption, subfig, xcolor}


\captionsetup{font=small,labelfont=bf}
\graphicspath{ {./images/} }

\makepagestyle{mypagestyle}
\copypagestyle{mypagestyle}{empty}
\makeoddhead{mypagestyle}{Nurtdinov Damir 31.05.2002\\d.nurtdinov@innopolis.university}{\quad}{{October 2022}\\Theoretical Mechanics: HW5}

\makeheadrule{mypagestyle}{\textwidth}{\normalrulethickness}
\makeoddfoot{mypagestyle}{}{\thepage}{}

\pagestyle{mypagestyle}

\usepackage{enumerate}

\usepackage[output-decimal-marker={,}]{siunitx}

\usepackage[hidelinks=true]{hyperref}

\title{Theoretical Mechanics HW5}
\author{Дамир Нуртдинов}
\date{September 2022}

\def\doubleunderline#1{\underline{\underline{#1}}}

\begin{document}

\section*{Task 1}
\subsection{Problem:}
The legend about the sniper.\\
1. Find $\alpha$ which is needed to shoot the officer.\\
2. What the max height of cargo ship can be to make this shot?\\
3. Find $\alpha$ taking into consideration air resistance: $F_c(v^2) = -kv\Vec{v}$\\
$L = 1500$m, $m = 13.6$g, $k=1.3 * 10^{-5}$, $v_0 = 870$m/s, $g = 9.81 m/s^2$
\subsection{Solution:}
\underline{Research object:} Bullet\\
\underline{Motion:} Translatory\\
\underline{Condition:}\\
Initial: $x_0 = 0$, $\dot x_0 = v_0$, $t=0$;\\
"final": $x_f = L$, $\dot x_f = v_0$, $t=?$;\\
"H": $x_H = L/2$, $\dot x_H = ?$, $t=?$;\\
\underline{Force analysis:}\\
For 1-2 questions: $m\Vec{g}$\\
For the 3rd question: $\Vec{F_c}$, $m\Vec{g}$\\
\underline{Solution:}\\
1. There 2 different $\alpha$ for the sniper to shot the officer. One of the angle can be found using school formula of ballistics :
\begin{equation}
    \begin{split}
        L = \frac{v^2_0 sin(2\alpha)}{g}\\
        \alpha = \arcsin(\frac{gL}{v^2_0})\\
        \alpha_1 \approx 0.556982 \degree\\
        \alpha_2 = (180 - \alpha_1*2)/2 \approx 89.443\degree
    \end{split}
\end{equation}
2. Max height of the ship is at the centre of river (middle of distance L) and also can be calculated using school formula:
\begin{equation}
    \begin{split}
        H = \frac{v^2_0 sin^2(\alpha)}{2g}\\
        \alpha_1: H_{max} \approx 3.6455583m\\
        \doubleunderline{\alpha_2: H_{max} \approx 38574.336m}\\
    \end{split}
\end{equation}
3. This task requires more efforts, I wrote the 2nd Newton's law and projections on coordinates.
\begin{equation}
    \begin{split}
        x: ma_x = -F_c cos(\alpha) = -k v^2\\
        y: ma_y = -mg-F_c sin(\alpha)
    \end{split}
\end{equation}
The equation above can be rewritten in differential form:\\
\begin{equation}
    \begin{cases}
    \frac{dV_x}{dt} = - \frac{k}{m}\sqrt{V^2_x + V^2_y}V_x\\
    \frac{dV_y}{dt} = - g - \frac{k}{m}\sqrt{V^2_x + V^2_y}V_y\\
\end{cases}
\end{equation}
Numerical calculation of the system of nonlinear differential equations gives:
\begin{equation}
    \begin{split}
        \doubleunderline{\alpha_1 \approx 1.86784\degree}\\
        \doubleunderline{\alpha_2 \approx 68.65180\degree}\\
    \end{split}
\end{equation}
All calculations can be found in colab:
Link: \url{https://colab.research.google.com/drive/1IQBCqrKOnkBtLquocZN67vyyrzIYXEyp?usp=sharing}\\
Graphs for the $\alpha_1$:\\
\includegraphics[width=14cm]{images/fc_first.png}\\
Graphs for the $\alpha_2$:\\
\includegraphics[width=14cm]{images/fc.png}\\

\section*{Task 2}
\subsection{Problem:}
A particle M is moving inside of the cylindrical channel of moving object A with radius $r$.\\
Determine the equation of relative motion $x=f(t)$. Find the pressure force the particle acting on the channel wall. I consider the case, when M starts its motion from the center of A.\\
1. Simulate the mechanism.\\
2. Show all the accelerations and forces.\\
3. Plot the $x(t)$ till point won't leave the channel.\\
4. plot $N(t)$. \\
$m = 0.02$, $\omega = \pi$, $r=0.5$\\
\subsection{Solution:}
\underline{Research object:} System of particle M and object A. \\
\underline{Motion:} A – rotational, M - translatory\\
\underline{Condition:}\\
Initial: $x_0 = 0$, $\dot x_0 = 0.4$, $t=0$;\\
"final": $x_f = r$, $\dot x_f = ?$, $t=?$;\\
\underline{Force analysis:}\\
known: G\\
unknown: N, $\Phi^c_e$,  $\Phi_{cor}$\\
\underline{Solution:}\\
1. The simulation is available here:\\
Link: \url{https://www.geogebra.org/m/wd7qdhks}\\
2. All forces and accelerations shown in simulation.\\
3. To find $x(t)$ I used Newton's Second law:
\begin{equation}
    \begin{split}
        m\Vec{a} = \Vec{\Phi^c_e} + \Vec{G} + \Vec{\Phi_{cor}} + \Vec{N}\\
        x: m\ddot x = \Phi^c_e + Gsin(\omega t) = m\omega^2x+mgsin(\omega t)\\
    \end{split}
\end{equation}
The full solution of this differential equation is:
\begin{equation}
    x = C_1 e^{\omega t} +  C_2 e^{-\omega t} + A sin(\omega t)+ B cos(\omega t)
\end{equation}
From (6) I got that $B=0$ and $A = -\frac{g}{2\omega^2}$\\
To calculate $C_1, C_2$ I substituted initial conditions and got:\\
\begin{equation}
\begin{cases}
        C_1 + C_2 = 0\\
        \omega(C_1-C_2)- \frac{g}{2\omega} = 0.4\\
\end{cases}
\begin{cases}
        C_2 = (-0.4 - \frac{g}{2\omega})\frac{1}{2\omega}\\
        C_1 = -C_2\\
\end{cases}
\end{equation}
The full equation of motion is:
\begin{equation}
    \doubleunderline{ x = C_1 e^{\omega t} +  C_2 e^{-\omega t} + A sin(\omega t)+ B cos(\omega t)}
\end{equation}\\
Plot is here:\\ 
\includegraphics[width=14cm]{images/task2x.png}\\
4. N was calculated with Newton's Second law but in direction which is perpendicular to $x$ of particle.\\
\begin{equation}
    \begin{split}
        0 = N + \Phi_cor - Gcos(\omega t)\\ 
        N =  2mV_{rel}\omega - mgcos(\omega t)\\
    \end{split}
\end{equation}
\includegraphics[width=14cm]{images/task2n.png}\\

\section*{Task 2}
\subsection{Problem:}
There are 3 weights connected with an ideal string. When 1 does down on $l$ = 1 meter the system shifts on distance S. Find the S\\
$m_1 = 20$ kg, $m_2 = 15$ kg, $m_3 = 10$ kg, $m = 100$ kg\\
\subsection{Solution:}
\underline{Research object:} System of 3 weights and prism ABC. \\
\underline{Motion:} 1,2,3, ABC – translatory\\
\underline{Condition:}\\
Initial: $x_0 = 0$, $\dot x_0 = 0$, $t=0$;\\
"final": $x_f = ?$, $\dot x_f = ?$, $t=?$;\\
\underline{Force analysis:}\\
Look at the picture:\\
\includegraphics[width=14cm]{images/ex3.png}\\
known: P1, P2, P3, P, N1, N2, N3, N\\
unknown: T1, T'2, T2, T3\\
\underline{Solution:}\\
Second Newton's law: 
\begin{equation}
    \begin{split}
        m\ddot x_c = 0 = P1 + P2 + P3+  P+ N1+ N2+ N3+ N +T1+ T'2+ T2+ T3 \\ 
        m\dot x_c = C_1, C_1 = 0\\
        mx_c = C_2, C_2 = 0\\
    \end{split}
\end{equation}
The center of mass of the system has not moved, but suppose the ABC moved to the left on S. 
Conditions for bodies:\\
initial: $x_1$, $x_2$, $x_3$, $x_{ABC}$\\
final: $x_1-S$, $x_2-S+l$, $x_3-s + lcos(60\degree)$, $x_{ABC}-S$\\
\begin{equation}
    \begin{split}
        \frac{m_1x_1 + m_2x_2+m3x_3+mx_{ABC}}{m_1+m_2+m_3+m} = \\\frac{m_1(x_1-s) + m_2(x_2-S+l)+m3(x_3-S+lcos(60\degree))+m(x_{ABC}-S)}{m_1+m_2+m_3+m}\\ 
        \doubleunderline{S = \frac{l(m_2 + m_3/2)}{m_1+m_2+m_3+m} = \approx 0.138 m}\\
    \end{split}
\end{equation}
\end{document}
