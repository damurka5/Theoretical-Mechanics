\documentclass[a4paper,11pt,oneside,article]{memoir}
% lidt marginer
\setlrmarginsandblock{3cm}{*}{1}
\setulmarginsandblock{3cm}{*}{1}
\setheadfoot{2cm}{\footskip}          
\checkandfixthelayout[nearest]

\setlength{\parindent}{0pt}

\usepackage[utf8]{inputenc}
\usepackage[danish]{babel}
\renewcommand\danishhyphenmins{22}
\usepackage[T1]{fontenc}
\usepackage{lmodern}
\usepackage{icomma}
\usepackage{gensymb}
\usepackage{bm, amsmath,mathdots, amssymb, mathtools}
\usepackage[normalem]{ulem}
\usepackage{array, booktabs, tabularx}
\usepackage{graphicx, caption, subfig, xcolor}


\captionsetup{font=small,labelfont=bf}
\graphicspath{ {./images/} }

\makepagestyle{mypagestyle}
\copypagestyle{mypagestyle}{empty}
\makeoddhead{mypagestyle}{Nurtdinov Damir 31.05.2002\\d.nurtdinov@innopolis.university}{\quad}{{October 2022}\\Theoretical Mechanics: Research HW2}

\makeheadrule{mypagestyle}{\textwidth}{\normalrulethickness}
\makeoddfoot{mypagestyle}{}{\thepage}{}

\pagestyle{mypagestyle}

\usepackage{enumerate}

\usepackage[output-decimal-marker={,}]{siunitx}

\usepackage[hidelinks=true]{hyperref}
\title{Theoretical Mechanics, Research HW2}
\author{Дамир Нуртдинов}
\date{October 2022}

\def\doubleunderline#1{\underline{\underline{#1}}}

\begin{document}

\section*{Task}
Yo-yo is fixed on the corner of the table. It will begin to unwind after
unfixing due to the gravity force. This yo-yo is designed in such a way that, having unwound, it starts to come back, but not to the full height. The process
of movement is an analog of a damping pendulum. The task is as follows. It is
necessary to find the number of oscillations before complete attenuation, as
well as the height by which the yo-yo rises each time. In order to solve this
problem, it is necessary to conduct an experiment and develop a mathematical
model. This model must be grounded. It is also necessary to compare the
simulation result and experimental results.

\subsection{Tools:}
1. Python (NumPy, MatPlotLib)\\
2. 2 cameras\\
3. Calipers\\
4. Scotch-tape

\subsection{Experiments:}
There was 15 experiments total: 5 different length and 3 experiments for each length.\\
Big thanks to Lev Kozlov who shared the data from the video-experiments in JSON format! Do not forget to visit his GitHub \url{https://github.com/lvjonok}
\begin{center}
\begin{tabular}{ | m{5em} | m{1cm}| m{2cm} | m{2cm} |m{2cm} |m{2cm} |} 
  \hline
   experiment& string-length (m) & oscillations in 1st repetition& oscillations in 2nd repetition& oscillations in 3rd repetition& Mean + std \\ 
  \hline
   No. 1 & 0.6 & 8 & 19 & 9 & 12 +/- 4.967\\ 
  \hline
  No. 2 & 0.5 & 9 & 14 & 9 & 10.667 +/- 2.357\\ 
  \hline
  No. 3 & 0.4 & 11 & 10 & 9 & 10 +/- 0.816\\ 
  \hline
  No. 4 & 0.3 & 10 & 10 & 10 & 10 +/- 0\\ 
  \hline
  No. 5 & 0.2 & 6 & 10 & 8 & 8 +/- 1.633\\ 
  \hline
\end{tabular}
\end{center}

Each experiment was recorded on 2 cameras: front-camera to track y-coordinate (left image) and bottom-camera to track rotations around string (right image)\\
\includegraphics[width=7cm]{images/pic1.jpeg}
\includegraphics[width=8cm]{images/pic2.jpeg}\\

Oscillation amplitude over first 5 oscillations (Mean for each experiment):\\
\begin{center}
\begin{tabular}{ | m{5em} | m{2cm}| m{2cm} | m{2cm} |m{2cm} |m{2cm} |} 
  \hline
   experiment& 1st oscillation (m) & 2nd oscillation (m) & 3rd oscillation (m)& 4th oscillation (m)& 5th oscillation (m) \\ 
  \hline
   No. 1 & 0.6 & 0.41 & 0.3 & 0.22 & 0.14\\ 
  \hline
  No. 2 & 0.5 & 0.35 & 0.235 & 0.19 & 0.135\\ 
  \hline
  No. 3 & 0.4 & 0.275 & 0.21 & 0.16 & 0.125\\ 
  \hline
  No. 4 & 0.3 & 0.27 & 0.14 & 0.215 & 0.195\\ 
  \hline
  No. 5 & 0.2 & 0.11 & 0.09 & 0.07 & 0.055\\ 
  \hline
\end{tabular}
\end{center}

\subsection{Mathematical Model:}
There are several ways to represent the motion of yo-yo. I tried to create model which describes yo-yo in 2 DoF. \\
I wrote the equation which describes kinetic and potential energy  of yo-yo, then I introduced a special term ($d$) which generalizes losses in kinetic energy due friction, resistance and etc. $n$ is a number of oscillation \\
\begin{equation}
    \begin{split}
        T_2 - T_1=\Pi_2-\Pi_1\\
        d^n(\frac{1}{2}mV^2 + \frac{1}{2}J\omega^2) = mgy(t)\\
        V = \dot y\\
        \omega = \frac{V}{r}\\
        J = m\rho^2
    \end{split}
\end{equation}
y-axis directed to the floor, $r$ - inner radius of yo-yo, $J$ - inertia of yo-yo, $\rho$ - inertia radius\\
The equation (1) describes only downward motion of yo-yo, here is the equation when the body moves upward.\\
\begin{equation}
    d^n(\frac{1}{2}mV^2 + \frac{1}{2}J\omega^2) = mg(L-y(t))\\
\end{equation}
These equations gives such trajectory ($d$ = 0.75, L = 0.6m)\\
\includegraphics[width=12cm]{images/pic3.png}\\

Lets try compare this model with experimental results:\\
\includegraphics[width=12cm]{images/pic4.png}\\

As you can see, there are differences in frequency, but the amplitude is precise enough. \\
I measured yo-yo parameters again and noticed that the inner radius changes significantly – it should cause a change in the frequency of oscillations. \\
\includegraphics[width=7cm]{images/pic5.jpeg}
\includegraphics[width=7cm]{images/pic6.jpeg}\\
The left picture above shows the inner radius at the very bottom point of trajectory, the right picture shows the inner radius at the very top of motion (L=0.6m). \\
Here is the table which represents relation between inner radius and starting length 

\begin{center}
\begin{tabular}{ | m{5em} | m{5em}| } 
  \hline
   Length (m)& max radius (mm) \\ 
  \hline
   0.6 & 15.3 \\ 
  \hline
  0.5 & 14.9 \\ 
  \hline
  0.4 & 13.2 \\ 
  \hline
  0.3 & 11.5 \\ 
  \hline
  0.2 & 10.9 \\ 
  \hline
\end{tabular}
\end{center}
\includegraphics[width=7cm]{images/pic7.png}\\
The graph above shows that the inner radius changes approximately linearly with change of length.\\
Lets use this information and upgrade equation (1), just introduce $r=r(y)$.
Simulations gives such results:\\
\includegraphics[width=7cm]{images/pic8.png}
\includegraphics[width=7cm]{images/pic9.png}\\
\includegraphics[width=7cm]{images/pic10.png}
\includegraphics[width=7cm]{images/pic11.png}\\
\includegraphics[width=7cm]{images/pic12.png}\\
As you can see below the model gives really good results in most of the cases for the first 4 oscillations.
Lets try to take into consideration rotation of yo-yo around the string. \\
My assumption about a cause of such rotations is gyroscopic effect.\\
\includegraphics[width=10cm]{images/pic13.png}\\
The goal is to find $\Omega$ and compare results with experiments. \\
Let denote $K_0 = J\omega$ as angular momentum of yo-yo and $M_0 = F_{tension} r$ as sum of all moment of tension force (suppose this is the only one force apllied to it). I just followed Alexander Maloletov's slides:\\
\begin{equation}
    \begin{split}
        \frac{dK_0}{dt} = M_0\\
        \frac{d \Vec{OB}}{dt} = V_B\\
        \frac{d \Vec{OB}}{dt} \textasciitilde M_0\\
        \Vec{V_B} = \Vec{\omega} \Vec{OB} \textasciitilde \Omega \times K_0\\
        \Omega \times K_0 = M_0\\
        \Omega = \frac{M_0}{J\omega}\\
        M_0 = F_{tension} l\\
        F_{tension} = m(g-\ddot y)
    \end{split}
\end{equation}
 Where $l$ is a lever arm for the tension force \\

Knowing $\Omega$ I found that yo-yo should rotate on about 10$\degree$ for the first oscillation, but it did not (1-2$\degree$ max).\\
\includegraphics[width=7cm]{images/pic14.png}
\includegraphics[width=7.5cm]{images/pic15.png}\\

As you can see, the gyroscopic effect does not converge with experiments due to naive assumption that the lever arm is always constant. Moreover, the thread is laid randomly, which is why the gyroscopic effect does not matter at all. But I at least tried to connect some not magical laws to rotations in 3D.\\
There is another reason – string (method of weaving is spiral). At the very bottom position yo-yo faces sharp hit - velocity changes its direction, acceleration is very high, and, therefore, the tension in string is very high and the net tries to unwind. This "unwind effect" maybe rotates yo-yo. 
\subsection{Conclusion:}
The model I provided works good for the first several oscillations of yo-yo. To minimize differences I tried to take into consideration more parameters, but still the model is not so complex. In real life yo-yo is affected by many forces, some of them are difficult to measure. Moreover, yo-yo is never used when it is fixed on some stick, we usually play with it, changing the height of end of rope. 

\end{document}
